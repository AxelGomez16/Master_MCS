\documentclass[12pt,a4paper]{article}

% ------------------------------
% LANGUAGE AND ENCODING
% ------------------------------
\usepackage[utf8]{inputenc}
\usepackage[T1]{fontenc}
\usepackage[spanish]{babel}
\usepackage{csquotes} % Recommended by biblatex with babel

% ------------------------------
% MATH AND SYMBOLS
% ------------------------------
\usepackage{amsmath, amssymb, amsfonts, bm, cancel, braket}

% ------------------------------
% PAGE AND MARGIN SETTINGS
% ------------------------------
\usepackage[left=2cm,right=2cm,top=2.5cm,bottom=2.5cm]{geometry}
\setlength{\headheight}{14.61pt} % Fix fancyhdr warning
\addtolength{\topmargin}{-2.61pt}

% ------------------------------
% HEADER & FOOTER
% ------------------------------
\usepackage{fancyhdr}
\pagestyle{fancy}
\fancyhf{}
\fancyhead[R]{\textit{\rightmark}}
\fancyfoot[C]{\thepage}
\renewcommand{\headrulewidth}{0.5pt}
\renewcommand{\footrulewidth}{0.5pt}

% ------------------------------
% GRAPHICS
% ------------------------------
\usepackage{graphicx}
\usepackage{float}
\usepackage{subcaption}
\usepackage{wrapfig}

% ------------------------------
% TABLES AND COLORS
% ------------------------------
\usepackage[table,xcdraw]{xcolor}
\usepackage{booktabs}
\AtBeginDocument{%
  \renewcommand{\tablename}{Tabla}%
  \renewcommand{\listtablename}{Índice de tablas}%
}

% ------------------------------
% CODE LISTINGS
% ------------------------------
\usepackage{listings}
\usepackage{textcomp}
\usepackage{color}

\definecolor{codegreen}{rgb}{0,0.6,0}
\definecolor{codegray}{rgb}{0.5,0.5,0.5}
\definecolor{codepurple}{rgb}{0.58,0,0.82}
\definecolor{backcolour}{rgb}{0.95,0.95,0.92}

\lstdefinestyle{mystyle}{
    backgroundcolor=\color{backcolour},
    commentstyle=\color{codegreen},
    keywordstyle=\color{magenta},
    numberstyle=\tiny\color{codegray},
    stringstyle=\color{codepurple},
    basicstyle=\ttfamily\small,
    breaklines=true,
    captionpos=b,
    keepspaces=true,
    numbers=left,
    numbersep=8pt,
    showspaces=false,
    showstringspaces=false,
    showtabs=false,
    tabsize=2,
    frame=single,
    rulecolor=\color{black}
}
\lstset{style=mystyle}

% ------------------------------
% HYPERLINKS
% ------------------------------
\usepackage{hyperref}
\hypersetup{
    colorlinks=true,
    linkcolor=blue,
    citecolor=blue,
    urlcolor=blue
}

% ------------------------------
% BIBLIOGRAPHY (APA 7)
% ------------------------------
%\usepackage[style=apa, backend=biber, language=spanish]{biblatex}
%\addbibresource{references.bib} % File should be named references.bib
\usepackage{apacite}


% ------------------------------
% CAPTIONS AND LIST NAMES
% ------------------------------
\usepackage{caption}
\renewcommand{\tablename}{Tabla}
\renewcommand{\figurename}{Figura}
\renewcommand{\lstlistingname}{Código}
\renewcommand{\lstlistlistingname}{Índice de Códigos}


% ------------------------------
% FUNCTIONS
% ------------------------------
% ------------------------------
% GRAPHICS
% ------------------------------
\usepackage{graphicx} % Make sure to include this in your preamble

% ------------------------------
% CODE LISTINGS
% ------------------------------
\usepackage{listings}
\usepackage{xcolor} % for colors (optional)
\usepackage{caption} % to style captions nicely
% -------------------------------------------------


% Define a new command \addFig that takes arguments: 

% #1 = image filename
% #2 = caption text
% #3 = width (e.g., 0.5\linewidth, 5cm, etc.)
\newcommand{\addFig}[3]{%
  \begin{figure}[htbp]
    \centering
    \includegraphics[width=#3]{#1}
    \caption{#2}
    \label{fig:#1}
  \end{figure}
}



% ------------------------------
% DOCUMENT BEGIN
% ------------------------------
\begin{document}

% ------------------------------
% TITLE PAGE
% ------------------------------
\begin{titlepage}
    \begin{center}
        \includegraphics[width=0.7\textwidth]{images/logo_uaq_info.png}
        \vspace{1cm}

        {\Large \textbf{Universidad Autónoma de Querétaro}}\\[0.5cm]
        {\large Facultad de Informática}\\[1cm]

        {\Large \textbf{Algoritmos Avanzados}}\\[0.5cm]
        {\large Dra. Diana Córdova Esparza}\\[1.5cm]

        {\LARGE \textbf{Práctica \#1}}\\[0.2cm]
        {\large \textit{Vectores y MATLAB}}\\[1cm]

        {\large Saúl Axel López Gómez}\\[0.3cm]
        {\large Agosto 2, 2025}
    \end{center}
\end{titlepage}

% ------------------------------
% CONTENTS
% ------------------------------
\tableofcontents
\listoffigures
\listoftables
\lstlistoflistings
\clearpage

% ------------------------------
% SECTIONS
% ------------------------------
\section{Introducción}
Aquí inicia la introducción del documento. Explica brevemente los objetivos de la práctica, la importancia del tema, y una visión general.

\section{Fundamentos teóricos}
Expón los conceptos que sustentan la práctica. Puedes incluir fórmulas, definiciones o diagramas. Por ejemplo, según \cite{2022}, el estudio de la óptica tiene aplicaciones fundamentales en telecomunicaciones.

\section{Material}
\begin{itemize}
    \item MATLAB R2023a
    \item PC con sistema operativo Windows 11
    \item Documentación oficial de MATLAB
\end{itemize}

\section{Metodología}
Aquí un ejemplo de figura: 


\addFig{images/logo_uaq_info.png}{This is the caption}{0.6\linewidth}

\begin{table}[htp]
\begin{center}
\caption[Factor, dimensiones y masa de la familia de satélites CubeSat]{Factor, dimensiones y masa de la familia de satélites CubeSat }\vspace{0.1in}
\label{cubedimen}
\begin{tabular}{clclclclc}
\hline
Factor               &  & \begin{tabular}[c]{@{}c@{}}X Dim\\ (cm)\end{tabular} &  & \begin{tabular}[c]{@{}c@{}}Y Dim\\ (cm)\end{tabular} &  & \begin{tabular}[c]{@{}c@{}}Z Dim\\ (cm)\end{tabular} &  & \begin{tabular}[c]{@{}c@{}}Masa\\ (kg)\end{tabular} \\ \hline
\multicolumn{1}{l}{} &  & \multicolumn{1}{l}{}                                 &  & \multicolumn{1}{l}{}                                 &  & \multicolumn{1}{l}{}                                 &  & \multicolumn{1}{l}{}                                \\
1U                   &  & 10                                                   &  & 10                                                   &  & 10                                                   &  & 1.33                                                \\
\multicolumn{1}{l}{} &  & \multicolumn{1}{l}{}                                 &  & \multicolumn{1}{l}{}                                 &  & \multicolumn{1}{l}{}                                 &  & \multicolumn{1}{l}{}                                \\
1.5U                 &  & 10                                                   &  & 10                                                   &  & 15                                                   &  & 1.55                                                \\
\multicolumn{1}{l}{} &  & \multicolumn{1}{l}{}                                 &  & \multicolumn{1}{l}{}                                 &  & \multicolumn{1}{l}{}                                 &  & \multicolumn{1}{l}{}                                \\
2U                   &  & 10                                                   &  & 10                                                   &  & 20                                                   &  & 2.66                                                \\
\multicolumn{1}{l}{} &  & \multicolumn{1}{l}{}                                 &  & \multicolumn{1}{l}{}                                 &  & \multicolumn{1}{l}{}                                 &  & \multicolumn{1}{l}{}                                \\
3U                   &  & 10                                                   &  & 10                                                   &  & 30                                                   &  & 4.00                                                \\
\multicolumn{1}{l}{} &  & \multicolumn{1}{l}{}                                 &  & \multicolumn{1}{l}{}                                 &  & \multicolumn{1}{l}{}                                 &  & \multicolumn{1}{l}{}                                \\
6U                   &  & 10                                                   &  & 20                                                   &  & 30                                                   &  & 8.00                                                \\
\multicolumn{1}{l}{} &  & \multicolumn{1}{l}{}                                 &  & \multicolumn{1}{l}{}                                 &  & \multicolumn{1}{l}{}                                 &  & \multicolumn{1}{l}{}                                \\
12U                  &  & 20                                                   &  & 20                                                   &  & 30                                                   &  & 12.00                                               \\ \hline
\end{tabular}
\end{center}
\end{table}

\section{Resultados}
Ejemplo de código en MATLAB con su cita:

\begin{lstlisting}[language=Matlab, caption={Vectorización de datos}]
x = 0:0.1:2*pi;
y = sin(x);
plot(x, y);
title('Seno de x');
xlabel('x');
ylabel('sin(x)');
\end{lstlisting}

\newpage



\section{Conclusiones}
Resumen de lo aprendido, dificultades encontradas, y reflexiones para futuras prácticas \cite{Blokhin2017}.

\newpage    
\bibliographystyle{apacite}
\bibliography{references/references.bib}

\end{document}





PONER CODIGOS: 
\lstset{inputencoding=utf8/latin1}
\lstinputlisting[language=Octave, caption=descripcion]{archivo.m}


PONER IMAGENES:
\begin{figure}[h]
\centering
\includegraphics[width=5cm]{NOMBRE DE LA IMAGEN}
\caption{DESCRIPCION}
\end{figure}



PONER IMAGENES DE LADITO
\begin{wrapfigure}[12]{r}{0.4\linewidth}
\centering
\includegraphics[width=5cm]{Escudo.jpg}   
\caption{Una imagen.}
\end{wrapfigure}


PONER DOS O MAS IMAGENES JUNTAS UNA A LA OTRA
\begin{figure}[htbp]
\centering
\subfigure[descripcion de la primer figura]{\includegraphics[width=5cm]{f2.png}}
\subfigure[descripcion de la segunda figura]{\includegraphics[width=5cm]{f3.png}}
\subfigure[descripcion de la tercer figura]{\includegraphics[width=5cm]{f3.png}}
\vspace{0.20in}
\caption{Leyenda} 
\end{figure}



EXPORTAR DATOS EN TXT A LATEX
\begin{center}
\begin{minipage}{9cm}
\VerbatimInput{datos1.txt}
\end{minipage}
\end{center}



TABLAS      https://www.tablesgenerator.com
\begin{table}[htp]
\begin{center}
\caption[Factor, dimensiones y masa de la familia de satélites CubeSat]{Factor, dimensiones y masa de la familia de satélites CubeSat }\vspace{0.1in}
\label{cubedimen}
\begin{tabular}{clclclclc}
\hline
Factor               &  & \begin{tabular}[c]{@{}c@{}}X Dim\\ (cm)\end{tabular} &  & \begin{tabular}[c]{@{}c@{}}Y Dim\\ (cm)\end{tabular} &  & \begin{tabular}[c]{@{}c@{}}Z Dim\\ (cm)\end{tabular} &  & \begin{tabular}[c]{@{}c@{}}Masa\\ (kg)\end{tabular} \\ \hline
\multicolumn{1}{l}{} &  & \multicolumn{1}{l}{}                                 &  & \multicolumn{1}{l}{}                                 &  & \multicolumn{1}{l}{}                                 &  & \multicolumn{1}{l}{}                                \\
1U                   &  & 10                                                   &  & 10                                                   &  & 10                                                   &  & 1.33                                                \\
\multicolumn{1}{l}{} &  & \multicolumn{1}{l}{}                                 &  & \multicolumn{1}{l}{}                                 &  & \multicolumn{1}{l}{}                                 &  & \multicolumn{1}{l}{}                                \\
1.5U                 &  & 10                                                   &  & 10                                                   &  & 15                                                   &  & 1.55                                                \\
\multicolumn{1}{l}{} &  & \multicolumn{1}{l}{}                                 &  & \multicolumn{1}{l}{}                                 &  & \multicolumn{1}{l}{}                                 &  & \multicolumn{1}{l}{}                                \\
2U                   &  & 10                                                   &  & 10                                                   &  & 20                                                   &  & 2.66                                                \\
\multicolumn{1}{l}{} &  & \multicolumn{1}{l}{}                                 &  & \multicolumn{1}{l}{}                                 &  & \multicolumn{1}{l}{}                                 &  & \multicolumn{1}{l}{}                                \\
3U                   &  & 10                                                   &  & 10                                                   &  & 30                                                   &  & 4.00                                                \\
\multicolumn{1}{l}{} &  & \multicolumn{1}{l}{}                                 &  & \multicolumn{1}{l}{}                                 &  & \multicolumn{1}{l}{}                                 &  & \multicolumn{1}{l}{}                                \\
6U                   &  & 10                                                   &  & 20                                                   &  & 30                                                   &  & 8.00                                                \\
\multicolumn{1}{l}{} &  & \multicolumn{1}{l}{}                                 &  & \multicolumn{1}{l}{}                                 &  & \multicolumn{1}{l}{}                                 &  & \multicolumn{1}{l}{}                                \\
12U                  &  & 20                                                   &  & 20                                                   &  & 30                                                   &  & 12.00                                               \\ \hline
\end{tabular}
\end{center}
\end{table}